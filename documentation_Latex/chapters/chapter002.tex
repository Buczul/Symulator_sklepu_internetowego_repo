\chapter{Opis założeń projektu}
\label{cha:założenia}

\section{Założenia projektu}
\label{sec:zalozenia}

Projekt to symulator sklepu internetowego "BookHaven" specjalizującego się w sprzedaży książek w różnych formatach: książek fizycznych, ebooków oraz audiobooków.
Głównym celem aplikacji jest odzwierciedlenie pełnego, realnego procesu korzystania ze sklepu internetowego, od rejestracji i logowania, po przeglądanie
oferty i składanie zamówień po stronie klienta, a także możliwość zarządzania produktami, zamówieniami i użytkownikami po stronie administratora.



% ------------------------
\section{Wymaganie funkcjonalne}
\label{sec:wymagania funkconajle}


Zarządanie użytkownikami musi obejmować rejestrację nowych użytkowników oraz logowanie na konto. 
Każdy użytkownik ma przypisaną rolę, która określa jakie działania będzie mógł on podejmować w aplikacji.
Administrator ma mieć możliwość usuwania istniejących użytkowników oraz dodawania nowych administratorów,
natomiast po stronie klienta dostępna jest opcja aktualizacji danych potrzebnych do wysyłki. 

Klient może przeglądać całą ofertę, ma dostęp do filtrowania oraz sortowania produktów w ofercie oraz dodawania ich do koszyka,
w którym można przeglądać wybrane produkty, usunąć je z niego, oraz złożyć zamówienie.

Administrator ma mieć narzędzia pozwalające na zarządzanie produktami: dodawanie, edycja i usuwanie.

Proste zarządanie zamówieniami również musi odbywać się po stronie administratora. Dostępne ma być przeglądanie wszystkich zamówień, gdzie widnieją
dane zamawiającego oraz produkty które zamówił, dodatkowo, bezpośrednio w aplikacji, administrator ma możliwość zmiany statusu zamówienia.

Aplikacja obsługuje możliwość zalogowania się jako gość, gdzie nie ma potrzeby logowania, natomiast ma on
mieć wówczas dostęp jedynie do przeglądania oferty.


%---------------------------------------------------------------------------

\section{Wymaganie niefunkcjonalne}
\label{sec:wymagania niefunkcjonalne}

Istotna jest wydajność aplikacji, krótki czas ładowania poszczególnych stron aplikacji, a także bezproblemowa obsługa
rozbudowanej bazy produktów i użytkowników.

Niezawodność projektu powinna być zapewniona przez obsługę wyjątków dla, możliwie wszystkich, funkcjonalości aplikacji oraz 
informowanie użytkownika o jego błędnych działaniach.

Graficzny interfejs użytkownika musi być prosty i intuicyjy. Poszczególne panele aplikacji powinny być przejyżste, tak aby
klienci, goście i administratorzy nie mieli problemu z nawigowaniem po nich. Oprócz tego interfejst graficzny ma wyglądać estetycznie, każda strona 
aplikacji nie powinna odbiegać wizualnie od pozostałych dzięki zostosowaniu wszędzie jednolitego białego tła oraz banera w kolorze \texttt{BA2F33} z nazwą sklepu.

Możliwa ma być również skalowalność aplikacji, obejmująca zwiększającą się bazę produktów, użytkowników i zamówień, a także możliwość późniejszej
rozbudowy systemu o dodatkowe funkcjonalność, między innymi dodanie nowych metod płatności lub implementacja przechowywania produktów z koszyka w bazie.