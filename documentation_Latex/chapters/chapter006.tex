\chapter{Podsumowanie}
\label{cha:podsumowanie}

Projekt "Symulator sklepu internetowego" został zrealizowany zgodnie z założeniami i wymaganiami 
określonymi na początku prac. Aplikacja umożliwia użytkownikom pełne korzystanie z funkcjonalności sklepu internetowego, w tym rejestrację, logowanie, przeglądanie oferty, dodawanie produktów do koszyka oraz składanie zamówień. Administratorzy mają możliwość zarządzania użytkownikami, produktami i zamówieniami, co zapewnia kompleksową obsługę systemu.

Interfejs graficzny został zaprojektowany w sposób przejrzysty i intuicyjny, co ułatwia nawigację zarówno klientom, jak i administratorom. Wykorzystanie bazy danych MySQL pozwoliło na efektywne przechowywanie i zarządzanie danymi, a zastosowanie podejścia obiektowego Javy oraz korzystanie z biblioteki SWING zapewniło wysoką wydajność i skalowalność aplikacji.

Projekt spełnił wszystkie założone cele, a także uwzględnił wymagania niefunkcjonalne, takie jak szybkość działania, niezawodność i estetyka interfejsu. Dalszy rozwój aplikacji może obejmować rozszerzenie funkcjonalności, np. dodanie nowych metod płatności czy tworzenie nowych kategorii produktów.

Podsumowując, projekt stanowi kompleksowe rozwiązanie symulujące działanie sklepu internetowego, które może służyć jako podstawa do dalszych prac rozwojowych.