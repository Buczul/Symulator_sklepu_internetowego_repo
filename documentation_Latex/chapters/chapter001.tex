\chapter{Streszczenie w języku polskim i angielskim}
\label{cha:Streszczenie}

\section{Streszczenie w języku polskim.}

Projekt pt. "Symulator sklepu internetowego" ma na celu stworzenie aplikacji imitującej działanie realnego sklepu internetowego specjalizującego 
się w sprzedaży książek w różnych formatach: fizycznych, ebooków i audiobooków. Aplikacja została napisana w języku Java z wykorzystaniem technologii 
takich jak Swing dla interfejsu graficznego oraz MySQL dla zarządzania bazą danych. Głównym celem projektu było odzwierciedlenie pełnego procesu obsługi 
klienta, od rejestracji i logowania, przez przeglądanie oferty i składanie zamówień, po zarządzanie produktami i użytkownikami przez administratora. 
Aplikacja uwzględnia również rolę gościa, który może przeglądać ofertę bez konieczności logowania. Projekt spełnia wymagania funkcjonalne i niefunkcjonalne, 
takie jak wydajność, niezawodność i intuicyjność interfejsu.

\section{Streszczenie w języku angielskim.}

The project titled "Online Store Simulator" aims to create an application that mimics the operation of a real online store specializing in 
the sale of books in various formats: physical books, ebooks, and audiobooks. The application was developed in Java using technologies such as
 Swing for the graphical interface and MySQL for database management. The main goal of the project was to reflect the complete customer service process, 
 from registration and login, through browsing the offer and placing orders, to product and user management by the administrator. The application also includes a
  guest role, allowing users to browse the offer without logging in. The project meets functional and non-functional requirements, such as performance, reliability,
   and interface intuitiveness.